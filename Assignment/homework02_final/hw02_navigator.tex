%\documentclass[12pt,a4]{article}

%\usepackage{amsmath}

%



\usepackage{graphicx,amsmath,amssymb,amsthm, boxedminipage,xcolor,amscd,amsbsy,latexsym,url,bm}

%\usepackage[lined,boxed]{algorithm2e}

\usepackage{algorithm}
\usepackage{algpseudocode}


\newtheorem{theorem}{Theorem}[section]
\newtheorem{proposition}[theorem]{Proposition}
\newtheorem{lemma}[theorem]{Lemma}
\newtheorem{corollary}[theorem]{Corollary}
\newtheorem{definition}[theorem]{Definition}

\newtheorem*{theorem*}{Theorem}
\newtheorem*{lemma*}{Lemma}
\newtheorem*{solution}{Solution}
\newtheorem*{proposition*}{Proposition}


\newtheorem{exercise}[theorem]{Exercise}
\newtheorem{exerciseD}[theorem]{*Exercise}
\newtheorem{exerciseDD}[theorem]{**Exercise}

\let\oldexercise\exercise
\renewcommand{\exercise}{\oldexercise\normalfont}

\let\oldexerciseD\exerciseD
\renewcommand{\exerciseD}{\oldexerciseD\normalfont}

%\let\oldexerciseD\exerciseD
%\renewcommand{\exerciseD}{\oldexerciseD\normalfont}

%\let\oldexerciseDD\exerciseDD
%\renewcommand{\exerciseDD}{\oldexerciseDD\normalfont}

\newcommand{\E}{\mathbb{E}}
%\newcommand{\nth}[1]{#1^{\textsuperscript{th}}}
\newcommand{\scalar}[2]{\ensuremath{\langle #1, #2\rangle}}
\newcommand{\floor}[1]{\left\lfloor #1 \right\rfloor}
\newcommand{\ceil}[1]{\left\lceil #1 \right\rceil}
\newcommand{\norm}[1]{\|#1\|}
\newcommand{\pfrac}[2]{\left(\frac{#1}{#2}\right)}
\newcommand{\nth}[1]{#1^{\textsuperscript{th}}}
\newcommand{\core}{\textnormal{core}}



\newif\ifsolution

\solutionfalse

\newcommand{\answer}[1]{
\ifsolution
{\color{blue} #1}
\else
\fi
}



\newcommand{\poly}{\textnormal{poly}}
\newcommand{\quasipol}{\textnormal{quasipol}}
\newcommand{\ssubexp}{\textnormal{stronglySubExp}}
\newcommand{\wsubexp}{\textnormal{weaklySubExp}}
\newcommand{\simplyexp}{\textnormal{E}}
\newcommand{\expo}{\textnormal{Exp}}



\newcommand{\N}{\mathbb{N}}
\newcommand{\nn}{\mathbb{N}_0^n}
\newcommand{\R}{\mathbb{R}}
\newcommand{\Z}{\mathbb{Z}}


\definecolor{darkgreen}{rgb}{0,0.6,0}

\date{}

\title{
\hbox{  Mathematical Foundations of Computer Science}
  \vspace{3mm}
{\normalsize CS 499,	Shanghai Jiaotong University,  Dominik Scheder\\
%\vspace{3mm}
Spring 2019}
}


\begin{document}

\maketitle

%\begin{quotation}
%  You are welcome to discuss the exercises in the discussion
%  forum. Please take them serious. Doing the exercises is as important
%  than watching the videos.
%
%  I intentionally included very challenging exercises and marked them
%  with one or two ``$*$''. No star means you should be able to solve
%  the exercises without big problems once you have understood
%  the material from the video lecture. One star means it requires 
%  significant additional thinking. Two stars means it is not 
%  unlikely that you will fail to solve them, even once you have understood
%  the material and thought a lot about the exercise. Don't feel bad
%  if you fail. Failure is part of learning.
%
%  This is the first time this course is online. Thus there might be mistakes
%  (typos or more serious conceptual mistakes) in the exercises. I will be 
%  grateful if you point them out to me!
%\end{quotation}

\documentclass[12pt,a4]{article}




\usepackage{graphicx,boxedminipage,xcolor}
\usepackage{amsmath,amscd,amsbsy,amssymb,latexsym,url,bm,amsthm}
%\usepackage[lined,boxed]{algorithm2e}

\usepackage{algorithm}
\usepackage{algpseudocode}


\newtheorem{theorem}{Theorem}[section]
\newtheorem{proposition}[theorem]{Proposition}
\newtheorem{lemma}[theorem]{Lemma}
\newtheorem{corollary}[theorem]{Corollary}
\newtheorem{definition}[theorem]{Definition}
\newtheorem*{solution}{Solution}

\newtheorem*{theorem*}{Theorem}
\newtheorem*{lemma*}{Lemma}
\newtheorem*{proposition*}{Proposition}


\newtheorem{exercise}[theorem]{Exercise}
\newtheorem{exerciseD}[theorem]{*Exercise}
\newtheorem{exerciseDD}[theorem]{**Exercise}

\let\oldexercise\exercise
\renewcommand{\exercise}{\oldexercise\normalfont}

%\let\oldexerciseD\exerciseD
%\renewcommand{\exerciseD}{\oldexerciseD\normalfont}

%\let\oldexerciseDD\exerciseDD
%\renewcommand{\exerciseDD}{\oldexerciseDD\normalfont}

\newcommand{\E}{\mathbb{E}}
%\newcommand{\nth}[1]{#1^{\textsuperscript{th}}}
\newcommand{\scalar}[2]{\ensuremath{\langle #1, #2\rangle}}
\newcommand{\floor}[1]{\left\lfloor #1 \right\rfloor}
\newcommand{\ceil}[1]{\left\lceil #1 \right\rceil}
\newcommand{\norm}[1]{\|#1\|}
\newcommand{\pfrac}[2]{\left(\frac{#1}{#2}\right)}
\newcommand{\nth}[1]{#1^{\textsuperscript{th}}}
\newcommand{\core}{\textnormal{core}}



\newif\ifsolution

\solutionfalse

\newcommand{\answer}[1]{
\ifsolution
{\color{blue} #1}
\else
\fi
}



\newcommand{\poly}{\textnormal{poly}}
\newcommand{\quasipol}{\textnormal{quasipol}}
\newcommand{\ssubexp}{\textnormal{stronglySubExp}}
\newcommand{\wsubexp}{\textnormal{weaklySubExp}}
\newcommand{\simplyexp}{\textnormal{E}}
\newcommand{\expo}{\textnormal{Exp}}



\newcommand{\N}{\mathbb{N}}
\newcommand{\nn}{\mathbb{N}_0^n}
\newcommand{\R}{\mathbb{R}}
\newcommand{\Z}{\mathbb{Z}}


\definecolor{darkgreen}{rgb}{0,0.6,0}


\date{}

\title{
\hbox{  Mathematical Foundations of Computer Science}
  \vspace{3mm}
{\normalsize CS 499,  Shanghai Jiaotong University,  Dominik Scheder\\
%\vspace{3mm}
Spring 2019}
}

\begin{document}

\maketitle


\setcounter{section}{1}

\begin{itemize}
 \item Homework assignment published on Thursday, 2019-03-07
 \item Submit questions or first draft solutions by Sunday, 2019-03-10,  by
 email to the TA and to me (dominik.scheder@gmail.com)
 \item We will discuss some problems on Monday, 2019-03-11.
 \item You will receive feedback by Wednesday, 2019-03-13.
 \item Revise your solution and hand in your final submission by Sunday, 2019-03-17.
\end{itemize}


\section{Fibonacci Numbers and Other Recurrences}

\subsection{Identities among the Fibonacci Numbers}


\begin{exercise}
    Prove the following identity:
    \begin{align*}
    \sum_{i=1}^{n} F_i = F_{n+2} - 1 \ .
    \end{align*}
    Actually, prove it twice:
    \begin{enumerate}
    \item Give an inductive proof.
    \item Give a {\em combinatorial} argument. Remember that $F_{n+2} = |A_n|$, where
    $A_n := \{ x \in \{0,1\}^n \ | \ x \textnormal{ does not contain 11} \}$. Find a way to partition
    $A_n$ into sets such that the identity above becomes obvious.
    \end{enumerate}
\end{exercise}

%==================================Begin 2.1=====================================
\begin{solution} \quad
       \begin{enumerate}
          \item We first give the inductive proof.
          
          \textbf{Basis Step:} For $n=0$, we have 
          $$
          \sum_{i=1}^0F_i=0=F_2-1.
          $$
        

         \textbf{Induction Hypothesis:} Assume that the statement is true for some $k\geq0$. Then we have $F_1+F_2+\cdots+F_k=F_{k+2}-1$.

         \textbf{Proof of Induction Step:} Remember that we have $F_{k+2}+F_{k+1}=F_{k+3}$ derived from the definition of the Fibonacci number. For $k+1$, we can derive that
         \begin{align*}
            \sum_{i=1}^{k+1}F_i=\sum_{i=1}^kF_i+F_{k+1}
            =F_{k+2}-1+F_{k+1}=F_{k+3}-1,
         \end{align*}
         which means that the statement holds true for $k+1$.

         \item We can see that the set $A_n$ can be  divided as 
         $$
         A_n=\{x\in\{ 0,1\}^n\ |\ 0y, y\in A_{n-1}\  \text{or}\ 10z,z\in A_{n-2} \}.
         $$ 
         And $A_{n-1},\cdots,A_2$ can further be divided recursively. Let us view the sequence from left to right, so $|A_{n-2}|$ equals to the number of such sequences in $A_n$ that ``1'' first appears at the very beginning. And $|A_{n-3}|$ equals to the number of such sequences in $A_n$ that ``1'' first appears at the second place. This pattern continues until we have that $|A_0|$ equals to the number of such sequences in $A_n$ that ``1'' first appears at the last but one place.

         In this case we have two sequences left in $A_n$, one sequence is where ``1'' first appears at the last place, the other is where there is only ``0''s. That means 
         $$
         \sum_{i=0}^{n-2}|A_i|=|A_n|-2=F_{n+2}-2.
         $$
          We also have that 
         $$
         \sum_{i=1}^nF_i=F_1+\sum_{i=2}^nF_i=F_1+\sum_{i=0}^{n-2}|A_i|,
         $$ 
         where $F_1$ equals $1$. Therefore
         $$
         \sum_{i=1}^nF_i=1+F_{n+2}-2=F_{n+2}-1.
         $$
       \end{enumerate}
    \end{solution}
%==================================End 2.1=====================================

\begin{exercise}
   Prove the following identity:
   \begin{align*}
      F_1 + F_3 + F_5 + \cdots + F_{2n+1} = F_{2n+2} \ .
   \end{align*}
   Again, give two proofs, one using induction on $n$ and one using a combinatorial argument 
   involving the sets $A_i$.
\end{exercise}

%==================================Begin 2.2=====================================
\begin{solution}\quad
   \begin{enumerate}
      \item Inductive proof.
      
      \textbf{Basis Step:} For $n=0$, we have 
      $$
      \sum_{i=0}^0F_{2i+1}=1=F_2.
      $$
    

     \textbf{Induction Hypothesis:} Assume that the statement is true for some $k\geq0$. Then we have $F_1+F_3+\cdots+F_{2k+1}=F_{2k+2}$.

     \textbf{Proof of Induction Step:} Remember that we have $F_{k+2}+F_{k+1}=F_{k+3}$ derived from the definition of the Fibonacci number. For $k+1$, we can derive that
     \begin{align*}
        \sum_{i=0}^{k+1}F_{2i+1}=\sum_{i=0}^kF_{2i+1}+F_{2k+3}
        =F_{2k+2}+F_{2k+3}=F_{2k+4},
     \end{align*}
     which means that the statement holds true for $k+1$.

     \item Similar with Exercise 2.1, which divides $A_n$ according to the number of continuous 0. In Exercise 2.2, we divides $A_n$ according to the number of continuous ``10".     
     $$
     A_n=\{x\in\{ 0,1\}^n\ |\ 0y, y\in A_{n-1}\  \text{or}\ 10z,z\in A_{n-2} \}.
     $$ 
     In last exercise, the subset `` $0y, y\in A_{n-1}$" is further divided to smaller subsets, while the subset `` $10z,z\in A_{n-2}$" is kept. For formula in Exercise 2.2, we will keep `` $0y, y\in A_{n-1}$" and divide `` $10z,z\in A_{n-2}$" to smaller subsets recursively.
     
     To be specific, `` $10z,z\in A_{n-2}$" can be divided as `` $100y, y\in A_{n-3}$" and `` $1010z,z\in A_{n-4}$". We keep `` $100y, y\in A_{n-3}$" and further divide `` $1010z,z\in A_{n-4}$", and so on. Finally, $A_n$ can be divided to subsets `` $0y, y\in A_{n-1}$",`` $100y, y\in A_{n-3}$",`` $10100y, y\in A_{n-5}$",$\cdots$.
     
  Therefore, $|A_{2n-1}|$ equals to the number of such sequences in $A_{2n}$ that no ``10'' appears at the very beginning. And $|A_{2n-3}|$ equals to the number of such sequences in $A_{2n}$ that one ``10'' first appears at the second place. This pattern continues until we have that $|A_1|$ equals to the number of such sequences in $A_2n$ that n-1 continuous ``10'' appears at the beginning.
  
     In this case we have one sequence left in $A_2n$ when the recurrence comes to the end, which is ``$1010\cdots101$". That means:
     $$
     \sum_{i=0}^{n-2}|A_{2i+1}|+1=\sum_{i=0}^{n-2}|A_{2i+1}|+F_1=|A_{2n}|.
     $$
  Therefore
     $$
     \sum_{i=0}^{n}F_{2i+1}=F_{2n+2}.
     $$
   \end{enumerate}
\end{solution}
%==================================End 2.2=====================================

\subsection{General Linear Recurrences}

For $a_1,\dots,a_k \geq 0$ we can consider the recursively defined numbers:

\begin{align*}
\begin{array}{lll}
    F_n  = &  f_n & \textnormal{if $n < k$} \\
    F_n =  &a_1 F_{n-1} + a_2 F_{n-2} + \cdots + a_k F_{n-k} &  \textnormal{if $n \geq k$}\ .
    \end{array}
\end{align*}
The values $f_0, \dots, f_{k-1}$ are the ``start values'' of the recurrence. For example, if we set $k=1$, $f_0=1$, and
$a_1 = 2$ then $F_n = 2^n$; setting $k=2$, $f_0=0$, $f_1=1$, and $a_2 = a_1 = 1$ yields the Fibonacci numbers.
As with the Fibonacci numbers, we can write the recursion in matrix-vector form:
\begin{align*}
    \begin{pmatrix} F_n \\ F_{n-1} \\ \vdots \\ F_{n-k+1}\end{pmatrix} &  =
    &
    \begin{pmatrix} a_1 & a_2 & a_3 & \cdots & a_{k-2} & a_{k-1} &  a_k \\
                                 1 & 0     &  0 & \cdots &0 &  0 & 0\\
                                  0 & 1    & 0 & \cdots &0 & 0& 0\\
                                  \vdots   & \vdots      &  \vdots & \cdots &\vdots &  \vdots & \vdots \\
                                  0  & 0  & 0  & \cdots & 0 & 1 & 0 
    \end{pmatrix} \cdot
     \begin{pmatrix}
     F_{n-1} \\ F_{n-2} \\ \vdots \\ F_{n-k}
     \end{pmatrix}
\end{align*}
Let us denote the matrix by $A$. 
\begin{exercise}
     Show that $\lambda$ is an eigenvalue of $A$ if and only if 
     \begin{align}
     \lambda^k = a_1  \lambda^{k-1} +a_2  \lambda^{k-2} + \cdots + a_{k-2} \lambda^2 + a_{k-1} \lambda + a_k \ . 
     \label{charpol}
     \end{align}
     For an eigenvalue $\lambda$, show what the corresponding eigenvector is.
     \textbf{Hint.} You can do this by computing $\det(A- \lambda I)$. But there is a simpler way by thinking
     directly in terms of what eigenvectors are.
\end{exercise}

%==================================Begin 2.3=====================================
\begin{proof} \item
  We assume that $\lambda$ is an eigenvalue and the corresponding eigenvector is $\boldsymbol x$. With the assumption, we can have and only have $A \boldsymbol x=\lambda \boldsymbol x$, in which $\boldsymbol x=(x_1,x_2,x_3,\cdots,x_n)$. Then we have $(A -\lambda E)\boldsymbol x=\boldsymbol 0$.

\begin{align*}
  \begin{pmatrix} a_1-\lambda & a_2 & a_3 & \cdots & a_{k-2} & a_{k-1} &  a_k \\
                               1 & -\lambda    &  0 & \cdots &0 &  0 & 0\\
                                0 & 1    & -\lambda & \cdots &0 & 0& 0\\
                                \vdots   & \vdots      &  \vdots & \cdots &\vdots &  \vdots & \vdots \\
                                0  & 0  & 0  & \cdots & 0 & 1 & -\lambda
  \end{pmatrix} \cdot
  \begin{pmatrix}
  x_{1} \\ x_{2} \\x_{3}\\ \vdots \\ x_{k}
  \end{pmatrix} =
  \begin{pmatrix}
  0 \\ 0 \\0\\ \vdots \\ 0
  \end{pmatrix}
\end{align*}

$$ \left\{
\begin{array}{lcl}
(a_{1}-\lambda)x_{1}+a_{2}x_{2}+\cdots+a_{k}x_{k}=0      \\
x_{1}-\lambda x_{2}=0\\
x_{2}-\lambda x_{3}=0\\
 \qquad \vdots\\
x_{k-1}-\lambda x_{k}=0

\end{array} \right. $$

Since $x_{1},x_{2},\cdots,x_{k} \ne 0$, we have$$\lambda=\frac{x_{1}}{x_{2}}=\cdots=\frac{x_{k-1}}{x_{k}}$$
So we can obtain
$$ \left\{
\begin{array}{lcl}

x_{1}=\lambda^{k-1} x_{k}\\
x_{2}=\lambda^{k-2} x_{k}\\
\qquad \vdots \\
x_{k-1}=\lambda^{1} x_{k}\\

\end{array} \right. $$

Then we have
$$[(a_{1}-\lambda)\lambda^{k-1}+a_{2}\lambda^{k-2}+\cdots+a_{k}]x_{k}=0$$

For $x_{k} \ne 0$, we obtain $$(a_{1}-\lambda)\lambda^{k-1}+a_{2}\lambda^{k-2}+\cdots+a_{k}=0$$

That is exactly
\begin{align}
\lambda^k = a_1  \lambda^{k-1} +a_2  \lambda^{k-2} + \cdots + a_{k-2} \lambda^2 + a_{k-1} \lambda + a_k \ .
\label{charpol}
\end{align}

The corresponding eigenvector is
\begin {align*}
\boldsymbol x=
\begin{pmatrix}
\lambda_{k-1} \\ \lambda_{k-2} \\\lambda_{k-3}\\ \vdots \\ \lambda_{0}
\end{pmatrix}
\end{align*}
\end{proof}
%==================================End 2.3=====================================

\begin{exercise}
   Recall that $a_1,\dots,a_k \geq 0$. Assume further that $a_1 + \cdots + a_k > 1$. Show that
   among the solutions to (\ref{charpol}), there is exactly one solution $\lambda_1$ with $\lambda_1 > 0$,
   and this $\lambda_1$ is actually greater than $1$.
\end{exercise}

%==================================Begin 2.4=====================================
\begin{proof} \quad
  \begin{enumerate}
    \item Prove that there exist one eigenvalue $\lambda_1$ with $\lambda_1 > 0$. \par
    Construct function:
    \begin{align*}
    f(\lambda) = \lambda^k - a_1  \lambda^{k-1} -a_2  \lambda^{k-2} - \cdots - a_{k-2} \lambda^2 - a_{k-1} \lambda - a_k.
    \end{align*}
    Substitute $1$ into $f(\lambda)$, then we have $f(1) = 1 - a_1 - \cdots - a_k < 0$. \
    Substitute $a_1 + \cdots + a_k$ into $f(\lambda)$,
    \begin{align*}
    f(\sum\limits_{i = 1}^ka_i) &= (\sum\limits_{i = 1}^ka_i)^k - a_1  (\sum\limits_{i = 1}^ka_i)^{k-1} - \cdots - a_{k-1} \sum\limits_{i = 1}^ka_i - a_k \\
    &= (a_1 + \cdots + a_k)(\sum\limits_{i = 1}^ka_i)^{k-1} - a_1  (\sum\limits_{i = 1}^ka_i)^{k-1} - \cdots - a_{k-1} \sum\limits_{i = 1}^ka_i - a_k \\
    &=a_2 (\sum\limits_{i = 1}^ka_i)^{k-2} (\sum\limits_{i = 1}^ka_i - 1) + a_3 (\sum\limits_{i = 1}^ka_i)^{k-3} ((\sum\limits_{i = 1}^ka_i)^2 - 1)  \\ 
    & + \cdots + a_k ((\sum\limits_{i = 1}^ka_i)^{k-1} - 1) > 0.
    \end{align*}
    Since $f(\lambda)$ is obviously continuous on $[1 , \sum\limits_{i = 1}^ka_i ]$, according to the zero point theorem, we have that there exists one solution $\lambda_1$ to $f(\lambda) = 0$ with $\lambda_1 > 1$.
    \item Prove that $\lambda_1$ is the unique positive root. \par
    Assume that there exists at least another positive root named $\lambda_2$. \\
    We have $p  \lambda_1 = \lambda_2$, with $p =  \frac{\lambda_2}{\lambda_1} > 0$.
    \begin{enumerate}
      \item $p > 1$, then we have:
      \begin{align*}
      f(\lambda_2) = f(p\lambda_1) &= (p\lambda_1)^k - a_1(p\lambda_1)^{k-1} - \cdots - a_k \\
      &= \lambda_1^k p^k - a_1 \lambda_1^{k-1} p^{k-1} - \cdots - a_k \\
      &< (\lambda_1^k - a_1 \lambda_1^{k-1} - \cdots - a_k) p^k = 0
      \end{align*}
      Therefore $f(\lambda_2) < 0$, which contradicts the assumption that $\lambda_2$ is one solution.
      \item $0 < p < 1$, then we have:
      \begin{align*}
      f(\lambda_2) = f(p\lambda_1) &= (p\lambda_1)^k - a_1(p\lambda_1)^{k-1} - \cdots - a_k \\
      &= \lambda_1^k p^k - a_1 \lambda_1^{k-1} p^{k-1} - \cdots - a_k \\
      &> (\lambda_1^k - a_1 \lambda_1^{k-1} - \cdots - a_k) p^k = 0
      \end{align*} 
      Therefore $f(\lambda_2) > 0$, which contradicts the assumption that $\lambda_2$ is one solution.
    \end{enumerate}
    We have derived a contradiction, which allows us to conclude that our original assumption is false.
  \end{enumerate}
\end{proof}
%==================================End 2.4=====================================


\end{document}
