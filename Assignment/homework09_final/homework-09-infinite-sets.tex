\documentclass[12pt,a4]{article}



\input{preamble}



\setcounter{section}{8}


\begin{center}
  \large\textbf{Group: navigator} 
\end{center}
\begin{center}
  \begin{tabular}{rl}
 Xu Huan  & 517021910724 \\
 Tianyao Shi     &     517021910623 \\
Chenxiao Yang    &    517021910540  \\
Jiaqi  Zeng      &     517021910882  \\
  \end{tabular}
\end{center}
\newpage



\section{Infinite Sets}

In the lecture (and the lecture notes) we have showed that $\N \times \N \cong \N$, i.e.,
there is a bijection $f: \N \times \N \rightarrow \N$. From this, and by induction, it follows
quite easily that $\N^k \cong \N$ for every $k$.

\begin{exercise}
   Consider $\N^*$, the set of all finite sequences of natural numbers, that is,
   $\N^* = \{\epsilon\} \cup \N \cup \N^2 \cup \N^3 \cup \dots$. Here,
   $\epsilon$ is the empty sequence. Show that $\N \cong \N^*$ by defining
   a bijection $\N \rightarrow \N^*$.
\end{exercise}


% -----------------------------------------------------------------------------------
\textbf{Solution.} As we know, there are infinite prime numbers. Given $k \in \N$, we can reflect it as the $k^{th}$ smallest prime number, which we denote as $p_k$. For example:
$$p_1 = 2,\quad p_2=3, \quad p_3= 5$$
We define the following reflection from $x\in \N $ to $ \N^*$.
$$\begin{aligned}
	&x := x+1\\
	\text{if } &x= 1\text{, then $x\Rightarrow \epsilon$}\\
	&x = p_1^{c_1} p_2^{c_2} \cdots p_r^{c_r}\text{ , where $c_r>0$ and $c_1, c_2,\cdots, c_{r-1}\geq 0$}\\
	\text{let } &y = (c_1, c_2, \cdots, c_{r-1}, c_r-1) \in \N^r\\
	&x\Rightarrow y
\end{aligned}
$$ 
We subtract $1$ from $c_r$ in case the last dimension of $y$ is $0$. It can help distinguish $(x_1,x_2)$ and $(x_1,x_2,0)$. For example:
$$83\Rightarrow 84\Rightarrow2^2 3^1 5^0 7^1 \Rightarrow (2,1,0,0)\in \N^4$$
$$(2,1,0)\Rightarrow 2^2 3^1 5^1 \Rightarrow 60\Rightarrow 59$$
It's easy to see that this conversion is bijection between $\N$ and $\N^*$.    
% -----------------------------------------------------------------------------------

\begin{exercise}
   Show that $R \cong R \times R$. \textbf{Hint:} Use the fact that 
   $R \cong \{0,1\}^{\N}$ and thus show that $\{0,1\}^{\N} \cong \{0,1\}^{\N} \times \{0,1\}^{\N}$.
\end{exercise}

\textbf{Proof.}  We first prove this lemma: \emph{For sets $A,B,C$ and $D$, if $A\cong C$, $B\cong D$, then $A\times B\cong C\times D$.}

Since $A\cong C$, $B\cong D$, there exists bijection $f: A\rightarrow C$ and $g: B\rightarrow D$. Let $h: A\times B\rightarrow C\times D$, $h(<x,y>)=<f(x),g(y)>$, then it's easy to see that $h$ is a bijection. Therefore $A\times B\cong C\times D$.

Then we show that  $\{0,1\}^{\N} \cong \{0,1\}^{\N} \times \{0,1\}^{\N}$. Suppose $x\in \{0,1\}^{\N}=(x_0x_1x_2x_3\cdots)$, let $f:\{0,1\}^{\N} \rightarrow \{0,1\}^{\N} \times \{0,1\}^{\N}$, $f(x)=(x_0x_2x_4\cdots,x_1x_3x_5\cdots)$.
\begin{enumerate}
   \item $f$ is an injection. $\forall x,y \in \{0,1\}^{\N_0}$ with $x\neq y$, $\exists i\in\N$ s.t. $x_i\neq y_i$. Without loss of generality, suppose $i$ is an odd number, then $f(x)=(x_0x_2x_4\cdots,x_1x_3x_5\cdots x_i\cdots)\neq(y_0y_2y_4\cdots,y_1y_3y_5\cdots y_i\cdots)=f(y)$.
   \item $f$ is a surjection. $\forall (x,y)\in  \{0,1\}^{\N} \times \{0,1\}^{\N}=(x_0x_1x_2\cdots,y_0y_1y_2\cdots)$, $\exists z=(x_0y_0x_1y_1x_2y_2\cdots) \in  \{0,1\}^{\N}$ s.t. $f(z)=(x,y)$.
\end{enumerate}
Therefore $f$ is a bijection, and $\{0,1\}^{\N} \cong \{0,1\}^{\N} \times \{0,1\}^{\N}$. Recall the fact that  $R \cong \{0,1\}^{\N}$, according to the lemma, we have $R \times R \cong \{0,1\}^{\N} \times \{0,1\}^{\N}$. Then according to the transitivity of $\cong$ relation, we have  $R \cong \{0,1\}^{\N} \cong \{0,1\}^{\N} \times \{0,1\}^{\N}\cong R \times R$.\qed

\begin{exercise}
  Consider $\R^{\N}$, the set of all infinite sequences $(r_1, r_2, r_3,\dots)$ of real numbers.
  Show that $\R \cong \R^{\N}$. \textbf{Hint:} Again, use the fact that $\R \cong \{0,1\}^{\N}$.
\end{exercise}

\textbf{Proof.} Remember the fact that $\R \cong \{0,1\}^{\N}\cong 2^{\N}$. That is to say, each real number is a function $f:\N\rightarrow\{0,1\}$, and $\R$ is the set of all functions from $\N\rightarrow\{0,1\}$. $\R^{\N}$ is therefroe a sequence of such functions.

Let function $g$ be that $g:\N\times\N\rightarrow\{0,1\}$, $\mathcal{G}:g(m,n)=f_n(m)$, where $f_n:\N\rightarrow\{0,1\}$ is a function as well as a real number, and the subscript $n$ means that it is the $n$-th element in the real number sequence. Let us denote the set of functions from $\N\times\N\rightarrow\{0,1\}$ as $A$, then $\mathcal{G}$ is a bijection between $A$ and $\R^{\N}$. 
\begin{enumerate}
   \item For two different functions $g_1,g_2$, $\exists (m,n)\in \N\times\N$ s.t. $g_1(m,n)\neq g_2(m,n)$, accordingly we have $f_n(m)_1\neq f_n(m)_2$, which means different real numbers as the $n$-th element in the sequence, leading to different sequences. Therefore $\mathcal{G}$ is injective.
   \item For any sequence in $\R^{\N}$, we can construct a function $g:\N\times\N\rightarrow\{0,1\}$, by following $g(m,n)=f_n(m)$. That is to say, $\mathcal{G}$ is surjective.
\end{enumerate}

Also recall that $\N\cong\N\times\N$, thus there is a bijection $h:\N\rightarrow\N\times\N$. Then there is a bijection from the set of functions $f:\N\rightarrow\{0,1\}$ to $A$, the set of functions $g:\N\times\N\rightarrow\{0,1\}$, the former of which is just $\R$ and the latter of which $\cong R^{\N}$. Therefore $\R\cong\R^{\N}$.\qed


\begin{exercise}
  Let $\mathcal{F}$ be the set of all {\em continuous} functions $f: \R \rightarrow \R$. Show that
  $\mathcal{F} \cong \R$.
\end{exercise}


\begin{proof}
	The cardinality of $\mathcal{F}$ is at least $|\R|$ because every real number corresponds to a constant function. \\
	Suppose $f: \R \rightarrow \R$ is a continuous function. Let $x \in \R$. Then there is a Cauchy sequence of rational numbers such that $\lim_{n \to \infty}q_{n}=x$. Continuity of $f$ means that $\lim_{n\to\infty}f(q_n) = f(\lim_{n\to\infty}q_n)=f(x).$ \\
	This implies that the values of $f$ at rational numbers already determine $f$. In other words, the mapping $\Phi$: $\mathcal{F} \rightarrow \mathbb{R}^{\mathbf{Q}}$ is an injection, which means $|\mathcal{F}| \le |\mathbb{R}^{\mathbf{Q}}|$. Notice that $|\mathbb{R}^{\mathbf{Q}}| = |{2^{\mathbf{N}}}^{\mathbf{N}}| = |2^{\mathbf{N} \times \mathbf{N}}| = |2^{\mathbf{N}}| = |\R|$. Hence, the cardinality of $\mathcal{F}$ is at most $|\R|$. \\
	By the Schröder-Bernstein Theorem, we have that $\mathcal{F} \cong \R$.
\end{proof}





Next, let us view $\{0,1\}^{\N}$ as a partial ordering: given two elements $\mathbf{a}, \mathbf{b} \in \{0,1\}^{\N}$,
that is, sequences $\mathbf{a} = (a_1,a_2,\dots)$ and $\mathbf{b} = (b_1,b_2,\dots)$, we define
$\mathbf{a} \leq \mathbf{b}$ if $a_i \leq b_i$ for all $i \in \N$. Clearly,
$(0,0,\dots)$ is the minimum element in this ordering and $(1,1,\dots)$ the maximum.\\

\begin{exercise}
   Give a countably infinite chain in $\{0,1\}^{\N}$. Remember that a set $A$ is countably infinite
   if $A \cong \N$.
\end{exercise}

\begin{proof}
  Assume $x \in \N$, we create a sequence $00\cdots 011\cdots 1$, in which there are $x$ 1's. In this way, $1 \rightarrow 2 \rightarrow 3 \cdots$ is corresponding to $00\cdots 001 \rightarrow 00\cdots 011 \rightarrow 00\cdots 0111 \rightarrow \cdots$ and this makes a countably infinite chain in $\{0,1\}^{\N}$.
\end{proof}

\begin{exercise}
   Find a countably infinite antichain in $\{0,1\}^{\N}$.
\end{exercise}

\begin{proof}
  Given $x \in \N$, we create a sequence $00\cdots 010\cdots 0$, in which 1 is in the last $x^{th}$ digit of this sequence. In this way, $1 \rightarrow 2 \rightarrow 3 \rightarrow \cdots$ is corresponding to $00\cdots 001 \rightarrow 00\cdots 010 \rightarrow 00\cdots 0100 \rightarrow \cdots$ and this makes a countably infinite antichain in $\{0,1\}^{\N}$.
\end{proof}

\begin{exercise}
   Find an uncountable antichain in $\{0,1\}^{\N}$. That is, an antichain $A$ with $A \cong \R$.
\end{exercise}

\begin{proof}
  For the number $0$, we corresponding create a sequence $100\cdots 00$.

  Now we exclude $0$ from the following steps. Given $a,b \in \N$ and $a,b$ are mutually prime and $a<b$, we consider four situations: (1)$+\frac{a}{b}$ (2) $-\frac{a}{b}$ (3) $+\frac{b}{a}$ (4) $-\frac{b}{a}$. \par

  First we create a sequence in which the $(a+1)^{th}$ and $(b+1)^{th}$ digits are 1. \par
  Second, we consider the following four digits after the latter 1 of the two. For case (1), the four digits are $1000$. For case (2), the four digits are $0100$. For case (3), the four digits are $0010$. For case (4), the four digits are $0001$.\par
  Third, set all other digits to $0$.

  For example, given $a=3$, $b=5$.\par
  $\frac{3}{5} \rightarrow 0001011000\cdots$\par
  $-\frac{3}{5} \rightarrow 0001010100\cdots$\par
  $\frac{5}{3} \rightarrow 00010100100\cdots$\par
  $-\frac{5}{3} \rightarrow 000101000100\cdots$\par

  In this way, we can construct an uncountable antichain in $\{0,1\}^{\N}$.
\end{proof}


\begin{exerciseDD}
   Find an uncountable chain in $\{0,1\}^{\N}$. That is, a chain $A$ with $A \cong \R$.
\end{exerciseDD}


\begin{exerciseDD}
   Find a set $X \subseteq 2^{\N}$ (that is, $X$ is a set of subsets of $\N$) such that
   (1) every $x \in X$ is an infinite subset of $\N$, (2) $x \cap y$ is finite whenever $x, y \in X$ 
   are distinct, (3) $X$ is uncountable.
\end{exerciseDD}





\end{document}



